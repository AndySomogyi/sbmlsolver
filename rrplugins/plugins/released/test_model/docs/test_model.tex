\chapter*{Test Model Plugin}
\setcounter{chapter}{1}
\emph{Do some testing}
\section{Introduction}
The purpose of the \emph{TestModel} plugin is to conveniently embed a SBML test model in a plugin. In addition, the plugin provides the user with simulated data, with and without applied artificial Gaussian noise. 

Currently no settings are exposed for the actual simulation of the test model.

The TestModel plugin depends on the AddNoise plugin.
 
\section{Plugin Parameters}
Table \ref{table:PluginProperties} lists available plugin property names, along with their data type and purpose.


\begin{table}[ht]
\centering % used for centering table
\begin{tabular}{l l p{7.5cm}} % centered columns (4 columns)

Parameter Name & Data Type & Purpose \\ [0.5ex] % inserts table 
%heading
\hline % inserts single horizontal line
Model         			& 	string 				& The actual test model, in XML format. \\
TestData      			& 	TelluriumData    	& Simulated data, using the TestModel as input and default RoadRunner Simulation values. \\
TestDataWithNoise    	& 	TelluriumData  	    & Simulated data, with applied noise. \\

\hline %inserts single line
\end{tabular}
\caption{Plugin Properties} 
\label{table:PluginProperties} 
\end{table}

\section{Plugin Events}
This plugin does not use any plugin events.


\section{The \texttt{execute()} function}
The \verb|execute()| function will generate simulated data, and simulated data with noise. The data will be available in the properties, TestData and TestDataWithNoise respectively. 

\section{Python examples}

\subsection{Usage of the TestModel plugin}
The python script below shows how to use the TestModel plugin. 

\begin{singlespace}
\lstinputlisting[label=test_plugin_header,caption={TestModel plugin example.},language=Python]{Examples/telExample.py}
\end{singlespace}

